\documentclass{beamer}


\title{Predstavitev 1.naloge}
\author{Kristian Tuvić}
\institute{Univerza v Ljubljani}
\date{7. november 2024}

\begin{document}

% Uvodna prosojnica
\begin{frame}
    \titlepage
\end{frame}

% Kazalo vsebine
\begin{frame}{Kazalo vsebine}
    \tableofcontents
\end{frame}

% Prosojnica 3: Opis vsebine datoteke naloga1_1.txt
\section{Vsebina datoteke naloga1\_1.txt}
\begin{frame}{Vsebina datoteke \texttt{naloga1\_1.txt}}
    \begin{itemize}
        \item V prvi vrstici: ime podatkovne datoteke
        \item V drugi vrstici: število vrstic in število podatkov v vsaki vrstici (100 vrstic in v vsaki 1 podatek)
        \item Na voljo so podatki o časovnih intervalih pri katerih je bila izmerjena moč
        \item MATLAB funkcija za branje datoteke: importdata
            \begin{itemize}
                \item Vhodni podatki: ime datoteke, ločilo med podatki, število začetnih vrstic, ki ga pri branju izpustimo
                \item Izhodni podatki: vektor vrednosti časov \( t \)
            \end{itemize}
    \end{itemize}
\end{frame}

% Prosojnica 4: Graf P(t)
\section{Graf funkcije \( P(t) \)}
\begin{frame}{Graf \( P(t) \)}
    \begin{figure}
        \centering
        \includegraphics[width=0.8\textwidth]{graf_Pt.png}
        \caption{Graf funkcije \( P(t) \), ki prikazuje odvisnost moči od časa}
    \end{figure}
\end{frame}

% Prosojnica 5: Trapezna funkcija za izračun integrala
\section{Trapezna formula}
\begin{frame}{Trapezna formula za izračun integrala}
    Trapezna formula za integral:
    \[
    \int_a^b f(x) \, dx \approx \frac{\Delta x}{2} \left( f(x_0) + 2f(x_1) + 2f(x_2) + \dots + 2f(x_{n-1}) + f(x_n) \right)
    \]
    
    Rezultat integrala iz naloge:
    \[
    \int_{t_{\text{min}}}^{t_{\text{max}}} P(t) \, dt \approx \text{17,1665}
    \]
\end{frame}

\end{document}
